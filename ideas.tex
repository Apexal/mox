\documentclass{article}[2017/07/09]

\title{Mox}
\author{Frank Matranga}
\date{July 2017 - }
\begin{document}
  \pagenumbering{gobble}
  \maketitle
  \newpage
  \pagenumbering{arabic}

  \tableofcontents
  \newpage

  \section{Introduction}
  \subsection{Name}
    \emph{Mox} is a Latin adverb meaning "soon".\\\\
    This alludes to my bad habit of coming up with ideas out of the blue, starting them, getting burned out within an hour, ditching them, and promising to come back to finish them "soon."
  \subsection{Purpose}
    This document serves as a repository of project ideas (most likely involving computer science) that I personally come up with or become interested in.
    These are projects I've thought of randomly, projects I've been thinking about for years, and everything in between.
    Ideas are listed with detailed descriptions of their theoretical use with limited reference to actual implementation as that can be left to actual projects if started.\\\\
    I also get to learn \LaTeX!
  \subsection{Motivation}
    On July 9th 2017, I decided that I needed a means of writing down all my potential project ideas somewhere in order to develop them (and not forget them!).
    I also happened to be learning \LaTeX\  at the same time so I put two and two together and decided to kill two birds with one stone.

  \section{Ideas}
  \paragraph{Ordering}
    Ideas will be ordered realistically in terms of feasibility, scope, and probability of them actually becoming a reality. Ideas deemed (by me) most likely to be attempted are listed first, and ideas least likely to become a reality will are listed last.
  \subsection{Character Directory}

  \subsection{Common Sense Directory}
    \subsubsection{What}
      A directory where I can submit some topic at the spur of a moment which I know less than I should about. I'd then be able to view added topics later on when I have the time to sit down and figure them out.
      Each topic will be able to store notes I write down as I familiarize myself with the topic. Topics should be heavily categorized to make them searchable and make it easy to analyze trends later on.
    \subsubsection{Why}
      Due to a combination of being very very selective in what information I paid attention to and a simple ignorance from lack of interest in common things
      from an early age, there are many real-world concepts I either barely grasp or have no understanding of whatsoever.
    \subsubsection{How}
      Since this is a simple idea, a simple SPA with Firebase as a backend might suffice.
    \subsubsection{Updates}
      I can just use Google Keep!
      
  \subsection{Daily Expense Tracker}
    \subsection{What
      A simple Android app that allows the user to day by day keep a list of cash spent in order to show overall stats and trends in simple numbers and graphs.
    \subsection{Why}
      Lately I've been spending cash way too heavily and over-generously and it is very difficult to stay aware of the amount I spend over longer time spans than a day. Currently I try to track my saved money every Sunday in a spreadsheet (which has revealed the disturbing downward trend) but this is too seldom and unweildy to update more frequently.
    \subsection{How}
      Using Android Studio, I can use this project as another way to learn Android development before the third trimester at school begins. Since the idea is simple I imagine it can be created with only a few Activities, and the most complicated parts will include learning to use databases (I must-know for my Android dev path) and perhaps displaying all collected data in charts and graphs. I imagine there are many, many tutorials for both of these areas I can readily access.
    \subsection{Updates}
    
 \end{document}
